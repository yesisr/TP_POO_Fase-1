\documentclass[a4paper, 12pt]{report}
\usepackage[portuguese]{babel}
\usepackage[utf8]{inputenc}
\usepackage[T1]{fontenc}
\usepackage{graphicx}
\usepackage{url}
\usepackage{float}
\usepackage{listings}
\usepackage{xcolor}
\usepackage{caption}
\usepackage{subcaption}
\usepackage{hyperref}

% Configuração para código C#
\definecolor{bluekeywords}{rgb}{0,0,1}
\definecolor{greencomments}{rgb}{0,0.5,0}
\definecolor{redstrings}{rgb}{0.64,0.08,0.08}

\lstset{
  inputencoding=utf8,
  extendedchars=true,
  literate={á}{{\'a}}1 {é}{{\'e}}1 {í}{{\'i}}1 {ó}{{\'o}}1 {ú}{{\'u}}1
           {â}{{\^a}}1 {ê}{{\^e}}1 {ô}{{\^o}}1
           {Á}{{\'A}}1 {É}{{\'E}}1 {Í}{{\'I}}1 {Ó}{{\'O}}1 {Ú}{{\'U}}1
           {ã}{{\~a}}1 {õ}{{\~o}}1 {Ã}{{\~A}}1 {Õ}{{\~O}}1
           {ç}{{\c{c}}}1 {Ç}{{\c{C}}}1
           {à}{{\`a}}1 {À}{{\`A}}1,
  language=[Sharp]C,
  captionpos=b,
  tabsize=4,
  showspaces=false,
  showstringspaces=false,
  basicstyle=\ttfamily\small,
  keywordstyle=\color{bluekeywords},
  commentstyle=\color{greencomments},
  stringstyle=\color{redstrings},
  frame=single,
  breaklines=true
}

\title{Relatório do Trabalho Prático de POO: \\ Sistema de Gestão de Obra de Construção Civil}
\author{Bruno Paiva \\ a31496@alunos.ipca.pt \\ \\ IPCA - Licenciatura de Engenharia de Sistemas Informáticos}
\date{Dezembro 2025}

\begin{document}

\maketitle

\begin{abstract}
Este relatório descreve o desenvolvimento da \textbf{Fase 1} do trabalho prático da Unidade Curricular de Programação Orientada a Objetos (POO), focado no tema \textbf{Gestão de Obra de Construção Civil}.
O objetivo principal é a aplicação do Paradigma Orientado a Objetos para modelar e gerir os recursos associados a uma obra, incluindo materiais, armazéns, mão de obra (própria e subcontratada), veículos e orçamentos.
O documento detalha a estrutura de classes, a definição de interfaces para garantir contratos claros e polimorfismo, e a aplicação de exceções customizadas para um tratamento de erros robusto.
\end{abstract}

\tableofcontents

\chapter{Introdução}
\section{Motivação e Tema}
O trabalho foi desenvolvido no contexto da UC de POO, com o propósito de aplicar conceitos de encapsulamento, herança, polimorfismo e abstração na resolução de problemas reais.
O tema escolhido, \textbf{Gestão de Obra de Construção Civil}, envolve a organização e controlo de custos e recursos de obras.
As entidades-chave modeladas incluem: a empresa (Business), a obra (ConstructionWork), materiais, armazéns, veículos, mão de obra e orçamentos.

\section{Objetivos da Solução}
Os principais objetivos atingidos nesta Fase 1 foram:
\begin{itemize}
    \item Identificação e modelação das entidades do domínio do problema.
    \item Definição de interfaces (\texttt{ICostable}, \texttt{IDescribable}, etc.) para padronização de comportamentos.
    \item Implementação das classes principais com encapsulamento adequado (propriedades e modificadores de acesso).
    \item Utilização de Coleções Genéricas (\texttt{List<T>}) para gestão dinâmica de dados.
    \item Criação de exceções personalizadas para validação de regras de negócio.
\end{itemize}

\chapter{Arquitetura e Estrutura da Solução}

\section{Estrutura do Projeto}
A solução encontra-se organizada no namespace \texttt{Projeto\_POO}, contendo todas as classes de definição de dados e lógica de negócio, bem como o ponto de entrada da aplicação (\texttt{Program.cs}).

\section{Interfaces Fundamentais}
As interfaces foram criadas para desacoplar as classes e permitir o tratamento genérico de objetos com características comuns:

\begin{itemize}
    \item \texttt{ICostable}: Define que um objeto possui um custo monetário (\texttt{decimal Cost}). Implementado por \texttt{Material}, \texttt{Labor}, \texttt{Vehicle} e \texttt{ConstructionWork}.
    \item \texttt{IDescribable}: Garante que a entidade possui uma descrição ou nome (\texttt{string Description}). Implementado por \texttt{Material}, \texttt{Storage}, \texttt{Labor}, \texttt{Budget} e \texttt{ConstructionWork}.
    \item \texttt{IIdentifiable}: Define um identificador numérico único (\texttt{int Code}). Implementado por \texttt{Material}.
    \item \texttt{IStorable}: Define que um item possui uma quantidade de armazenamento (\texttt{int Quantity}). Implementado por \texttt{StorageItem}.
    \item \texttt{IDateable}: Garante que a entidade possui uma data associada (\texttt{DateTime Date}). Implementado por \texttt{Budget}.
\end{itemize}

\section{Classes e Relações}
A arquitetura do sistema baseia-se nas seguintes entidades:

\subsection{Business (Empresa)}
A classe de topo que representa a empresa de construção. Gere o nome da empresa e mantém uma lista de obras (\texttt{ConstructionWork}).

\subsection{ConstructionWork (Obra)}
A classe central e agregadora. Representa uma obra específica e implementa \texttt{IDescribable} e \texttt{ICostable}.
Esta classe agrega listas de todos os recursos necessários:
\begin{itemize}
    \item \texttt{List<Storage>} (Armazéns)
    \item \texttt{List<Labor>} (Mão de Obra)
    \item \texttt{List<Vehicle>} (Veículos disponíveis)
    \item \texttt{List<VehicleUsage>} (Registos de uso de veículos)
    \item \texttt{List<Budget>} (Orçamentos)
\end{itemize}

\subsection{Recursos Materiais}
\begin{itemize}
    \item \textbf{Material}: Define as características intrínsecas de um produto (Código, Descrição, Custo Unitário). Implementa \texttt{IIdentifiable}, \texttt{IDescribable} e \texttt{ICostable}.
    \item \textbf{Storage}: Representa um armazém físico. Implementa \texttt{IDescribable} e contém uma lista de itens.
    \item \textbf{StorageItem}: Relaciona um \texttt{Material} com uma quantidade específica (\texttt{Quantity}). Implementa \texttt{IStorable}.
\end{itemize}

\subsection{Recursos Humanos e Máquinas}
\begin{itemize}
    \item \textbf{Labor}: Representa a mão de obra, distinguindo entre própria e subcontratada através de um booleano (\texttt{Subcontracted}). Implementa \texttt{IDescribable} e \texttt{ICostable}.
    \item \textbf{Vehicle}: Define um equipamento ou viatura (Matrícula, Modelo, Custo/Hora). Implementa \texttt{ICostable}.
    \item \textbf{VehicleUsage}: Regista a utilização efetiva de um veículo na obra. Armazena o veículo associado e as horas de uso (arredondadas para cima).
\end{itemize}

\subsection{Gestão Financeira}
\begin{itemize}
    \item \textbf{Budget}: Representa registos de orçamento com data, descrição e valor. Implementa \texttt{IDateable} e \texttt{IDescribable}.
\end{itemize}

\chapter{Detalhes de Implementação}

\section{Encapsulamento}
Todos os atributos das classes são privados (\texttt{private}), sendo o acesso feito exclusivamente através de propriedades públicas (\texttt{Properties}). Isso permite validações futuras nos \texttt{setters}, como visto na classe \texttt{VehicleUsage}, onde as horas são arredondadas automaticamente:

\begin{lstlisting}
public decimal Hours
{
    get => _hours;
    set => _hours = Math.Ceiling(value);
}
\end{lstlisting}

\section{Exceções Customizadas}
Para garantir a integridade dos dados e um tratamento de erros específico, foram criadas classes de exceção no ficheiro \texttt{custom\_exceptions.cs}:

\begin{itemize}
    \item \texttt{InvalidQuantityException}: Para erros em quantidades (e.g., negativas).
    \item \texttt{MaterialNotFoundException}: Quando se tenta aceder a um material inexistente.
    \item \texttt{StorageNotFoundException}: Para erros na localização de armazéns.
    \item \texttt{VehicleNotFoundException} e \texttt{VehicleExistsException}: Para gestão da frota de veículos.
    \item \texttt{ConstructionWorkNotFoundException}: Para validar a seleção da obra ativa.
\end{itemize}


\chapter{Conclusão}
A Fase 1 do trabalho prático permitiu estabelecer o modelo do sistema de Gestão de Obras. A estrutura de classes criada respeita os princípios de Programação Orientada a Objetos, com uma clara separação de responsabilidades e uso intensivo de interfaces.
A solução atual suporta a definição de todos os recursos necessários (materiais, mão de obra, veículos) e está preparada para a implementação da lógica de negócio mais complexa e persistência de dados na Fase 2.

\section{Trabalho Futuro (Fase 2)}
Para a próxima fase, os próximos passos incluem:
\begin{itemize}
    \item \textbf{Implementação Final:} Refinar a lógica de negócio e as validações.
    \item \textbf{Aplicação Demonstradora:} Implementar os métodos no projeto \texttt{Projeto\_POO} para exercer as funcionalidades pedidas.
    \item \textbf{Testes Unitários:} Implementar testes para garantir uma cobertura mínima de 50\% do código.
    \item \textbf{Persistência de Dados:} Adicionar a funcionalidade de guardar e carregar a obra em ficheiros binários.
\end{itemize}

\section{Link do Repositório GitHub}
Poderá aceder ao código-fonte completo através do seguinte link: 
\newline\url{https://github.com/yesisr/TP_POO_Fase-1}

\end{document}